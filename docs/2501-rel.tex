\documentclass[a4paper,12pt]{report}

\usepackage[italian]{babel}
\usepackage{hyperref}
\usepackage[utf8]{inputenc}
\usepackage{float}
\usepackage[table,xcdraw]{xcolor}
\usepackage{longtable}
\usepackage{graphicx}
\usepackage{todonotes}

\title{\textbf{CineRadar}\\Progetto di Basi di dati\\\textit{Università di Bologna}}
\author{Martin Tomassi; 0001077737\\Francesco Pazzaglia; 0001077423\\Luca Casadei; 0001069237}
\date{\today}

\begin{document}
	\maketitle
	\tableofcontents
	\chapter{Analisi dei requisiti}
	\section{Introduzione}
	Il progetto consiste nella realizzazione di un applicativo per la condivisione e la recensione di elementi multimediali quali film e serie TV.
	\section{Intervista}
	È richiesto un sistema che consenta all'utenza di accedere al portale di condivisione di film e serie TV e poter recensirne uno o più di interesse, nel dettaglio, un film è un elemento multimediale unico, mentre una serie TV è composta da più stagioni, composte a loro volta da un certo numero di episodi. È necessario che l'utente si registri inserendo i propri dati che verranno salvaguardati, tra cui username, password, nome, cognome, data di nascita per controllare i limiti di età sui film. Un utente deve avere la possibilità di aggiungere nuovi film o serie nel sistema, ma non in maniera diretta, la richiesta deve prima essere approvata da un altro tipo di utente con privilegi di amministratore del sistema. L'amministratore si occuperà di aggiungere film alla piattaforma con frequenza settimanale. Sulle recensioni di altri utenti, un utente può esprimere un parere di utilità che andrà a classificare la recensione in una sorta di "classifica delle recensioni più utili". Quando si va a memorizzare un film o una serie TV bisogna inserire il titolo, uno o più autori, la descrizione, la durata, l'anno di rilascio ed eventuale casa produttrice. Gli utenti devono essere in grado di vedere tutti gli elementi multimediali presenti in base a dei filtri sul genere, categoria, autori e recensioni, sempre a patto che i risultati ottenuti rispettino i limiti di età standard dei film/serie (\textit{over 18, over 16} etc\dots). Quando un utente vede un film della lista, deve poterlo segnare come \textit{"già visto"} per poi poterlo recensire se lo desidera, mentre una stagione di una serie TV è considerata "vista" se tutti i suoi episodi sono stati contrassegnati come visti. Dato che si vogliono anche incentivare le famiglie, è richiesta una funzionalità che renda possibile inserire un'età e visualizzare i film/serie la cui visione è consentita (per esempio per capire se un film/serie è adatto per il proprio figlio). Un utente registrato all'interno dell'applicativo può ricevere, in base alla quantità e qualità delle sue recensioni, dei coupon da utilizzare all'interno di un cinema che ha erogato tale sconto, dopo aver ottenuto la loro tessera da affiliato. Il coupon può essere utilizzato per un elemento multimediale o genere specifico entro la sua data di scadenza. Sarà compito dei cinema stabilire quale impiegato avrà la facoltà di erogare le tessere. Per tenere traccia delle strutture che hanno emesso delle tessere vengono memorizzati i cinema attraverso un codice numerico, un indirizzo ed un nome. All'interno di ogni cinema è presente uno o più addetti registratori che sono in grado di registrare nuove tessere per il cinema al quale afferiscono.
	Vi devono essere anche delle piccole classifiche per incentivare gli utenti ad effettuare recensioni, per esempio una classifica degli utenti con il maggior numero di film/serie visualizzati o recensiti. Vi è inoltre una statistica sui film/serie inseriti, per esempio sarà possibile ottenere il genere di film/serie con il maggior numero di visualizzazioni complessive. Oltre che sui film ci devono essere delle statistiche su chi i film/serie li fa, quindi per esempio il regista che ha girato il film/serie con le recensioni migliori e gli eventuali attori che hanno partecipato al cast. L'amministratore può attribuire dei premi in base alla classifica degli utenti migliori (in base alle recensioni più utili), effettuerà anche atti di moderazione sugli utenti registrati, ad esempio, se un utente effettua troppe recensioni al di sotto della soglia di utilità potrà essere rimosso o bloccato dall'effettuare l'accesso al sistema.
	\section{Rielaborazione del testo}
	\subsection{Obiettivi finali:}
	È richiesto un sistema che consenta all'utenza di accedere al portale di condivisione di \textcolor{orange}{\underline{film}} e \textcolor{orange}{\underline{serie TV}} per poterne recensire uno o più di interesse.\\
	Vi devono essere anche delle piccole \textcolor{orange}{\underline{classifiche}} per incentivare gli utenti ad effettuare \textcolor{orange}{\underline{recensioni}}, per esempio una classifica degli utenti con il maggior numero di film visualizzati o recensiti.\\ 
	Vi è inoltre una statistica sui film inseriti, per esempio sarà possibile ottenere il \textcolor{orange}{\underline{genere}} di film con il maggior numero di visualizzazioni complessive. Inoltre è richiesta anche una statistica sugli \textcolor{orange}{\underline{attori}} e i \textcolor{orange}{\underline{registi}} che hanno prodotto i film inseriti, esempio, l'attore con più comparse all'interno dei film inseriti.
	Un utente registrato sull'applicativo può ricevere, in base alla quantità e qualità delle sue recensioni, dei coupon da utilizzare all'interno di un cinema che ha erogato tale sconto, dopo aver ottenuto la loro tessera da affiliato. Il coupon può essere utilizzato per un film o genere specifico entro la sua data di scadenza.
	\subsection{Funzionalità lato utente:}
	È necessario che l'\textcolor{orange}{\underline{utente}} si registri inserendo i propri dati che verranno salvaguardati, tra cui username, password, nome, cognome, data di nascita per controllare i limiti di età sui film. Un utente deve avere la possibilità di aggiungere nuovi \textcolor{orange}{\underline{film}} attraverso una richiesta al sistema, essa deve prima essere approvata da un altro tipo di utente con privilegi di \textcolor{orange}{\underline{amministratore}} del sistema.\\
	Quando un utente vede un film della lista, deve poterlo segnare come \textcolor{orange}{\underline{"Visto"}} per poterlo recensire successivamente se lo desidera. Dato che si vogliono anche incentivare le \textcolor{orange}{\underline{famiglie}}, è richiesta una funzionalità che renda possibile inserire un'età e visualizzare i film/serie la cui visione è consentita (per esempio per capire se un film è adatto per il proprio figlio).\\
	Gli utenti devono essere in grado di vedere tutti i film/serie presenti in base a dei \textcolor{orange}{\underline{filtri}} sul genere, categoria, autori e recensioni, sempre a patto che i risultati ottenuti rispettino i \textcolor{orange}{\underline{limiti di età}} standard dei film/serie (\textit{over 18, over 16} etc\dots).
	\subsection{Funzionalità lato amministrativo:}
	L'\textcolor{orange}{\underline{amministratore}} si occuperà di aggiungere film/serie alla piattaforma con frequenza settimanale. Sulle recensioni di altri utenti, un utente può esprimere un parere di utilità che andrà a classificare la recensione in una sorta di "classifica delle recensioni più utili".\\
	Quando si va a memorizzare un film/serie bisogna inserire il titolo, uno o più autori, la descrizione, la durata, l'anno di rilascio ed eventuale casa produttrice.
	L'amministratore può attribuire dei \textcolor{orange}{\underline{premi}} in base alla classifica degli utenti migliori (in base alle recensioni più utili), effettuerà anche atti di moderazione sugli utenti registrati, ad esempio, se un utente effettua troppe recensioni al di sotto della soglia di utilità potrà essere rimosso o bloccato dall'effettuare l'accesso al sistema.
	\subsection{Funzionalità lato registratore:}
	All'interno di ogni cinema è presente uno o più addetti \textcolor{orange}{\underline{registratori}} che sono in grado di registrare nuove tessere per il cinema al quale afferiscono.
	
	\subsection{Termini da chiarire}\label{ss:terminologia}
	\begin{itemize}
		\item "Utente" $\longrightarrow$ Un utilizzatore dell'applicativo che si registra (o che accede) alla piattaforma ed ha come compito principale la visione, scrittura e valutazione di recensioni.
		\item "Visto" $\longrightarrow$ Un utente può spuntare una casella "visto" se ha visto effettivamente il film o un episodio di una serie.
		\item "Filtro" $\longrightarrow$ Un filtro viene applicato sulla ricerca che può fare un utente sulla lista di film/serie, questo può riguardare l'autore, il titolo, le recensioni dell'utenza etc\dots
		\item "Classifica" $\longrightarrow$ Una lista visualizzabile sull'applicativo in base a parametri scelti; il concetto non viene esplicitato nello schema concettuale ma sarà realizzato attraverso delle query.
		\item "Famiglia" $\longrightarrow$ Un utente genitore può effettuare una sorta di controllo parentale inserendo un'età in un'apposita casella e visualizzando tutti i film/serie che rispettano il limite di età inserito.
	\end{itemize}
	\section{Estrazione dei dati del testo rielaborato}
	\subsection{Estrazione dati sugli utenti}
	\textbf{Utente} $\longrightarrow$ Utente che si registra sull'applicativo\\Successivamente verranno elencati i dati da dover memorizzare.
	\begin{itemize}
		\item Username
		\item Password
		\item Nome
		\item Cognome
		\item Data di nascita
		\item Targa premio (opzionale)
	\end{itemize}
	\textbf{Amministratore} $\longrightarrow$ Utente privilegiato che effettua operazioni di moderazione e inserimento dati. Dato che nell'intervista non emergono i dati da memorizzare per l'amministratore, si presuppone che vi sia un contatto per poterlo raggiungere, oltre che alle credenziali.
	\begin{itemize}
		\item Username
		\item Password
		\item Nome
		\item Cognome
		\item Telefono
	\end{itemize}
	\subsection{Estrazione dati sui film/serie e produttori}
	\textbf{Film / Serie TV} $\longrightarrow$ Elementi multimediali su cui gli utenti registrati possono apporre la propria visualizzazione ed effettuare recensioni in seguito.
	\begin{itemize}
		\item Data di rilascio
		\item Titolo
		\item Descrizione
		\item Genere
		\item Limite di età
	\end{itemize}
	\textbf{Attori} $\longrightarrow$ Questi vengono memorizzati per stilare la classifica visualizzabile, ed esplicitare delle preferenze. Fanno parte del cast di un film.
	\begin{itemize}
		\item Nome
		\item Cognome
		\item Nome d'arte
		\item Data di nascita
	\end{itemize}
	\textbf{Registi} $\longrightarrow$ Sono coloro che dirigono il cast.
	\begin{itemize}
		\item Nome
		\item Cognome
		\item Data di nascita
		\item Debutto carriera
	\end{itemize}
	\subsection{Elenco delle azioni}
	\subsubsection{Utente}
	\begin{itemize}
		\item Registrarsi sulla piattaforma.
		\item Accedere alla piattaforma.
		\item Scegliere le proprie categorie di preferenza.
		\item Visualizzare i film/serie in base ai filtri scelti e ai limiti di età.
		\item Data un età inferiore alla propria, ottenere la lista dei film che sono visualizzabili con quell'età.
		\item Richiedere all'amministratore l'aggiunta di un film / serie non presente in elenco.
		\item Contrassegnare come visualizzato un film / serie.
		\item Recensire i film contrassegnati come visualizzati.
		\item Dare una valutazione di utilità alle recensioni degli altri utenti.
		\item Ricercare dei film in base al genere o all'autore.
		\item Visualizzare le classifiche degli autori e dei generi.
	\end{itemize}
	\subsubsection{Amministratore}
	\begin{itemize}
		\item Ottenere le statistiche degli utenti registrati alla piattaforma per poter effettuare attività di moderazione, tra cui: 
		\begin{itemize}
			\item (S)Bloccare o eliminare l'accesso ad un utente alla piattaforma se ha effettuato troppe recensioni che stanno sotto alla soglia di utilità.
			\item Premiare l'utente in cima alla classifica delle recensioni più utili (il premio è una targhetta che comparirà accanto al nome).
		\end{itemize}
		\item Aggiungere film / serie alla lista, compresi quelli che sono stati richiesti dagli utenti.
		\item Inserire nuovi autori o registi da associare a dei film.
	\end{itemize}
	\subsubsection{Registratore}
	\begin{itemize}
		\item Aggiungere nuove tessere per il cinema al quale afferisce.
	\end{itemize}
	\chapter{Progettazione concettuale}
	\section{Strategia bottom-up}
	\subsection{Utenza}
	Rappresentazione di una sottoparte dello schema ER che riguarda la gestione dell'utenza, in particolare, è stata pensata una generalizzazione dell'entità \textbf{utilizzatore} con due sotto-entità \textbf{amministratore} e \textbf{utente} di cui dobbiamo memorizzare elementi diversi. La generalizzazione è di tipo \textit{totale} e \textit{esclusiva}, questo perché l'amministratore ha un tipo di account diverso da quello di semplice utente, se l'amministratore vuole effettuare le operazioni di un utente normale deve registrarsi con un altro account.
	\begin{figure}[H]
		\centering
		\includegraphics[width=300pt]{ER/utenza.png}
		\caption{Schema ER dell'utenza.}
	\end{figure}
	\subsection{Multimedia}
	Qui viene descritto il concetto di \textbf{multimedia}(riferimento nella sezione \ref{ss:terminologia}), che si estende tramite generalizzazione alla rappresentazione di due sotto entità: \textbf{film} e \textbf{serie tv}. Un multimedia può essere associato a diversi \textbf{generi}, \textit{(quindi possono esistere, ad esempio, film/serie che sono sia horror che commedie)}, e l'inserimento dei generi è indipendente dall'esistenza dei multimedia, quindi si è optato per la cardinalità \textbf{0-N}. La principale differenza tra un film e una serie TV sta nel fatto che una serie può essere composta da più stagioni, ognuna con diversi episodi, mentre un film è un'unica narrazione senza interruzioni.
	\begin{figure}[H]
		\centering
		\includegraphics[width=300pt]{ER/multimedia.png}
		\caption{Schema ER dei multimedia.}
	\end{figure}
	\subsection{Cast}
	I membri del cast da tracciare possono essere \textbf{registi} o \textbf{attori}; degli attori ci interessa in particolare un nome d'arte, mentre dei registi la data di debutto della carriera per effettuare delle statistiche. La generalizzazione in questo caso è \textit{totale} e \textit{sovrapposta}, questo significa che nel dominio del nostro problema non si vogliono tracciare altri membri oltre che a questi due, ed è sovrapposta perché possono esserci casi in cui un regista è anche un attore.
	\begin{figure}[H]
		\centering
		\includegraphics[width=300pt]{ER/cast.png}
		\caption{Schema ER dei membri del cast.}
	\end{figure}
	\subsection{Recensione}
	Una recensione è suddivisa in più sezioni(ad esempio: trama, colonna sonora, etc\dots) e per ciascuna di esse viene assegnato un voto. Una sezione, invece, può esistere anche senza il concetto di recensione. 
	Il concetto di recensione si suddivide in due sottoconcetti, mediante una generalizzazione: recfilm (recensione di film) e recserie (recensione di serie TV). Entrambi sono identificati dall'utente che ha redatto la recensione, tuttavia recfilm è identificato anche dal film trattato, mentre recserie è identificata anche dalla serie televisiva.
	\begin{figure}[H]
		\centering
		\includegraphics[width=300pt]{ER/recensione.png}
		\caption{Schema ER della recensione.}
	\end{figure}
	\subsection{Promo}
	Una promozione è un'offerta limitata da una scadenza specifica, viene definita come un'istanza di un modello promozionale. Tale \textit{template} è caratterizzato da una percentuale di sconto, che gli utenti possono usufruire. La promo \textit{template} è suddivisa in due sotto entità in maniera totale ed esclusiva:
	\begin{itemize}
			\item Singolo: valido per uno e un solo film.
			\item Multiplo: valido per uno o più generi di film.
	\end{itemize}
	\begin{figure}[H]
		\centering
		\includegraphics[width=270pt]{ER/promo.png}
		\caption{Schema ER della promo.}
	\end{figure}
	\subsection{Relazione tra utenza e recensioni}
	Un utente può scrivere più di una recensione su vari film o serie tv, esplicitando il titolo, la descrizione della recensione e la sua valutazione. Quando viene scritta una recensione vengono scelti dei voti, ciascuno per ogni sezione selezionata dall'utente.
	\begin{figure}[H]
		\centering
		\includegraphics[width=300pt]{ER/utenzarecensione.png}
		\caption{Schema ER della relazione tra utenza e recensioni.}
	\end{figure}
	\subsection{Relazione tra utenza e cinema}
	Un \textbf{cinema} afferisce a più registratori, mentre ciascuno di essi è a disposizione solo su di un cinema in particolare. Il \textbf{registratore} è considerato un \textbf{utilizzatore}, quindi è compreso nella generalizzazione. 
	\begin{figure}[H]
		\centering
		\includegraphics[width=300pt]{ER/utenzacinema.png}
		\caption{Schema ER della relazione tra utenza e cinema.}
	\end{figure}
	\subsection{Relazione tra utenza e tessera}
	Gli utenti hanno la possibilità di iscriversi a una o più \textbf{tessere}. Una tessera è inserita da un \textbf{registratore}. Ciascuna tessera è caratterizzata da un numero univoco, che la identifica insieme all'utente tesserato, e da una data di rinnovo. 
	\begin{figure}[H]
		\centering
		\includegraphics[width=300pt]{ER/utenzatessera.png}
		\caption{Schema ER della relazione tra utenza e tessera.}
	\end{figure}
	\subsection{Relazione tra multimedia e membri del cast}
	Un \textbf{cast} può essere diretto da uno e un solo \textbf{regista} mentre possono prendere parte all'interno dello stesso uno o più \textbf{attori}. Queste due entità sono raggruppate tramite la generalizzazione "membro cast" che descrive un qualsiasi membro del cast cinematografico che ha presto parte nel film stesso. 
	Il cast è specifico per ogni singolo film, mentre può variare da stagione a stagione all'interno di una serie TV.
	\begin{figure}[H]
		\centering
		\includegraphics[width=350pt]{ER/multimediacast.png}
		\caption{Schema ER della relazione tra multimedia e membri del cast}
	\end{figure}
	\subsection{Relazione tra multimedia e promo}
	Un multimedia può essere oggetto di offerte promozionali sia direttamente che tramite il suo genere. Il modello di promozione viene suddiviso in singolo e multiplo attraverso una generalizzazione, distinguendo il primo che è specifico del multimedia stesso, mentre il secondo si riferisce al genere.
	\begin{figure}[H]
		\centering
		\includegraphics[width=300pt]{ER/multimediapromo.png}
		\caption{Schema ER della relazione tra multimedia e promo.}
	\end{figure}
	\subsection{Relazione tra multimedia, cast e promo}
	Di un cast vengono memorizzati i membri che, mediante una generalizzazione, sono suddivisi in regista e attori. Il cast è associato a un film particolare o a una stagione di una serie TV. Queste due entità sono aggregate dal concetto di multimedia, il quale, come precedentemente descritto, è collegato al sistema di offerte promozionali.
	\begin{figure}[H]
		\centering
		\includegraphics[width=300pt]{ER/multimediacastpromo.png}
		\caption{Schema ER della relazione tra multimedia, cast e promo.}
	\end{figure}
	\subsection{Relazione tra utenza, cinema, tessera e recensione}
	Una tessera, inserita da un registratore per il cinema per cui lavora, può essere utilizzata da un utente che è stato premiato perché risultante particolarmente attivo sulla piattaforma. Questi utenti possono usufruire di vantaggi quali sconti e offerte per mezzo della tessera stessa. Nello specifico una tessera è valida solo per un cinema, quello di afferenza del registratore.
	Inoltre includiamo il collegamento che vi è tra l'utenza e la recensione, che è stata precedentemente descritta.
	\begin{figure}[H]
		\centering
		\includegraphics[width=450pt]{ER/utenzacinematesserarecensione.png}
		\caption{Schema ER della relazione tra utenza, cinema, tessera e recensione.}
	\end{figure}
	\subsection{Schema completo}
	Quello che segue rappresenta lo schema derivato dalla composizione di tutte le sotto-componenti descritte precedentemente.
	\begin{figure}[H]
		\centering
		\includegraphics[width=430pt]{ER/ercompletosx.png}
		\caption{Schema ER completo \textbf{1/2}.}
	\end{figure}
	\begin{figure}[H]
		\centering
		\includegraphics[width=430pt]{ER/ercompletodx.png}
		\caption{Schema ER completo \textbf{2/2}.}
	\end{figure}
	\chapter{Progettazione Logica}
	\section{Stima del volume dei dati}
	\begin{table}[H]
		\centering
		\begin{tabular}{|lll|}
			\hline
			\rowcolor[HTML]{FFCE93} 
			\multicolumn{3}{|l|}{\cellcolor[HTML]{FFCE93}Film} \\ \hline
			\rowcolor[HTML]{CBCEFB} 
			Concetto             & Costrutto         & Volume         \\ \hline
			MULTIMEDIA           & E                 & 12000          \\ \hline
			FILM          		 & E                 & 9500           \\ \hline
			SERIE          		 & E                 & 2500           \\ \hline
			GENERE				 & E				 & 13			  \\ \hline
			categorizzazione	 & R				 & 12000		  \\ \hline
			visualizzazionefilm	 & R				 & 5000		  	  \\ \hline
		\end{tabular}
	\end{table}
	\begin{table}[H]
		\centering
		\begin{tabular}{|lll|}
			\hline
			\rowcolor[HTML]{FFCE93} 
			\multicolumn{3}{|l|}{\cellcolor[HTML]{FFCE93}Cast} \\ \hline
			\rowcolor[HTML]{CBCEFB} 
			Concetto             & Costrutto         & Volume         \\ \hline
			ATTORE               & E                 & 5200           \\ \hline
			REGISTA              & E                 & 600	          \\ \hline
			CAST              	 & E                 & 12000	      \\ \hline
			attorecast           & R                 & 12000            \\ \hline
			registacast          & R                 & 12000            \\ \hline
		\end{tabular}
	\end{table}
	\begin{table}[H]
		\centering
		\begin{tabular}{|lll|}
			\hline
			\rowcolor[HTML]{FFCE93} 
			\multicolumn{3}{|l|}{\cellcolor[HTML]{FFCE93}Serie TV} \\ \hline
			\rowcolor[HTML]{CBCEFB} 
			Concetto              & Costrutto        & Volume        \\ \hline
			STAGIONE              & E                & 25000       	 \\ \hline
			EPISODIO		   	  & E                & 30000000      \\ \hline
			compstagione      	  & R                & 30000000   	 \\ \hline
			compserie             & R                & 25000       	 \\ \hline
			visualizzazioneep	  & R				 & 15000000	  	 \\ \hline
		\end{tabular}
	\end{table}
	\begin{table}[H]
		\centering
		\begin{tabular}{|lll|}
			\hline
			\rowcolor[HTML]{FFCE93} 
			\multicolumn{3}{|l|}{\cellcolor[HTML]{FFCE93}Utilizzatori} \\ \hline
			\rowcolor[HTML]{CBCEFB} 
			Concetto               & Costrutto         & Volume        \\ \hline
			UTILIZZATORE           & E                 & 10000         \\ \hline
			AMMINISTRATORE         & E                 & 10            \\ \hline
			REGISTRATORE           & E                 & 30            \\ \hline
			UTENTE                 & E                 & 9960          \\ \hline
		\end{tabular}
	\end{table}
	\begin{table}[H]
		\centering
		\begin{tabular}{|lll|}
			\hline
			\rowcolor[HTML]{FFCE93} 
			\multicolumn{3}{|l|}{\cellcolor[HTML]{FFCE93}Cinema e registratori di tessere} \\ \hline
			\rowcolor[HTML]{CBCEFB} 
			Concetto                     & Costrutto                & Volume               \\ \hline
			CINEMA                       & E                        & 25                   \\ \hline
			TESSERA                      & E                        & 3500                 \\ \hline
			appartenenza                 & R                        & 3500                 \\ \hline
			tesseramento                 & R                        & 3500                 \\ \hline
			afferenza                    & R                        & 30                   \\ \hline
		\end{tabular}
	\end{table}
	\begin{table}[H]
		\centering
		\begin{tabular}{|lll|}
			\hline
			\rowcolor[HTML]{FFCE93} 
			\multicolumn{3}{|l|}{\cellcolor[HTML]{FFCE93}Coupon e premi su tessera} \\ \hline
			\rowcolor[HTML]{CBCEFB} 
			Concetto                   & Costrutto             & Volume             \\ \hline
			PROMO                      & E                     & 200                \\ \hline
			TEMPLATEPROMO              & E                     & 500         		\\ \hline
			MULTIPLO           		   & E                     & 25               	\\ \hline
			SINGOLO              	   & E                     & 475                \\ \hline
			premitessera               & R                     & 1500               \\ \hline
			validità                   & R                     & 200                \\ \hline
			couponsugenere             & R                     & 50                 \\ \hline
			couponsufilm               & R                     & 475                \\ \hline
		\end{tabular}
	\end{table}
	\begin{table}[H]
		\centering
		\begin{tabular}{|lll|}
			\hline
			\rowcolor[HTML]{FFCE93} 
			\multicolumn{3}{|l|}{\cellcolor[HTML]{FFCE93}Recensioni} \\ \hline
			\rowcolor[HTML]{CBCEFB} 
			Concetto              & Costrutto        & Volume        \\ \hline
			RECENSIONE            & E                & 1200000       \\ \hline
			SEZIONE				  & E				 & 5			 \\ \hline
			recensionefilm        & R                & 950000        \\ \hline
			recensioneserie       & R                & 250000         \\ \hline
			valutazione           & R                & 115000        \\ \hline
			valutazionesezione    & R				 & 1400000		 \\ \hline
			scritturarecensioneserie             & R                & 250000         \\ \hline
			scritturarecensioneserie             & R                & 950000         \\ \hline
		\end{tabular}
	\end{table}
	\section{Operazioni e loro frequenza}
	\subsection{Operazioni dell'utenza}
	\begin{longtable}[H]{|c|c|>{\columncolor[HTML]{FFFFC7}}c |c|}
		\hline
		\cellcolor[HTML]{ECF4FF}Numero &
		\cellcolor[HTML]{ECF4FF}Operazione &
		\cellcolor[HTML]{ECF4FF}Frequenza / gg &
		\cellcolor[HTML]{ECF4FF}Dettagli \\ \hline
		\endfirsthead
		%
		\endhead
		%
		1 &
		\begin{tabular}[c]{@{}c@{}}Registrazione di \\ un nuovo utente.\end{tabular} &
		2 &
		\\ \hline
		2 &
		Accesso alla piattaforma. &
		10 &
		\\ \hline
		3.1 &
		Scelta delle preferenze. &
		2 &
		Solo in fase di registrazione \\ \hline
		3.2 &
		\begin{tabular}[c]{@{}c@{}}Aggiornamento \\ delle preferenze.\end{tabular} &
		1 &
		\begin{tabular}[c]{@{}c@{}}Tra tutti gli utenti \\ che accedano in un \\ giorno, è plausibile \\ che pochi aggiornino \\ le proprie preferenze.\end{tabular} \\ \hline
		4.1 &
		\begin{tabular}[c]{@{}c@{}}Visualizzare tutto \\ l'elenco dei film.\end{tabular} &
		9 &
		\begin{tabular}[c]{@{}c@{}}Generalmente chi \\ effettua l'accesso ha\\ quella intenzione.\end{tabular} \\ \hline
		4.2 &
		\begin{tabular}[c]{@{}c@{}}Visualizzare l'elenco \\ dei film in base\\ ad un'età scelta.\end{tabular} &
		4 &
		\begin{tabular}[c]{@{}c@{}}Meno frequenza \\ rispetto all'operazione\\ precedente, \\ i genitori registrati che\\ accedono ed usano \\ questa funzione sono\\ meno.\end{tabular} \\ \hline
		5 &
		\begin{tabular}[c]{@{}c@{}}Contrassegnare \\ come "visualizzato" \\ un film.\end{tabular} &
		14 &
		\begin{tabular}[c]{@{}c@{}}Considerando che in \\ genere chi accede\\ all'applicativo e vede \\ la lista dei film,\\ scrive una recensione \\ su almeno un film,\\ implica che l'abbia \\ precedentemente\\ visualizzato. \\ Potrebbe essere \\ che venga contrassegnato \\ più di un film.\end{tabular} \\ \hline
		6 &
		Recensire un film. &
		10 &
		\begin{tabular}[c]{@{}c@{}}Dei film contrassegnati \\ in un giorno solo\\ alcuni verranno recensiti.\end{tabular} \\ \hline
		7.1 &
		\begin{tabular}[c]{@{}c@{}}Visualizzare le \\ recensioni di un film.\end{tabular} &
		25 &
		\\ \hline
		7.2 &
		\begin{tabular}[c]{@{}c@{}}Dare una valutazione \\ di utilità ad una\\ recensione di un altro \\ utente su un film.\end{tabular} &
		40 &
		\begin{tabular}[c]{@{}c@{}}Su alcune recensioni \\ che si visualizzano\\ si può dare una \\ valutazione di utilità,\\ quindi anche su più \\ recensioni dello stesso\\ film.\end{tabular} \\ \hline
		8.1 &
		\begin{tabular}[c]{@{}c@{}}Visualizzare una \\ classifica dei generi\\ più visualizzati.\end{tabular} &
		6 &
		\\ \hline
		8.2 &
		\begin{tabular}[c]{@{}c@{}}Visualizzare una \\ classifica dei registi in\\ base alla media \\ delle recensioni.\end{tabular} &
		6 &
		\\ \hline
	\end{longtable}
	\subsection{Operazioni di amministrazione}
	\begin{longtable}[H]{|c|c|>{\columncolor[HTML]{FFFFC7}}c |c|}
		\hline
		\cellcolor[HTML]{ECF4FF}Numero &
		\cellcolor[HTML]{ECF4FF}Operazione &
		\cellcolor[HTML]{ECF4FF}Frequenza &
		\cellcolor[HTML]{ECF4FF}Dettagli \\ \hline
		\endfirsthead
		%
		\endhead
		%
		9.1 &
		\begin{tabular}[c]{@{}c@{}}Reperimento della\\ classifica degli \\ utenti con la media\\ delle valutazioni di\\ utilità sulle proprie\\ recensioni peggiore.\end{tabular} &
		1 / mese &
		\\ \hline
		9.2 &
		\begin{tabular}[c]{@{}c@{}}Come 9.1 ma è la\\ media delle recensioni\\ migliori.\end{tabular} &
		1 / settimana &
		\begin{tabular}[c]{@{}c@{}}La premiazione degli\\ utenti con dei coupon\\ è settimanale.\end{tabular} \\ \hline
		10 &
		\begin{tabular}[c]{@{}c@{}}Assegnamento di coupon\\ ai primi 5 utenti tesserati \\ che si trovano in cima alla\\ classifica stilata (vedi 9.2)\end{tabular} &
		5 / settimana &
		\\ \hline
		11.1 &
		\begin{tabular}[c]{@{}c@{}}Aggiunta di un nuovo \\ film alla piattaforma.\end{tabular} &
		4 / giorno &
		\begin{tabular}[c]{@{}c@{}}Escono cira 2000 film \\ all'anno in tutto il mondo, \\ supponendo che si \\ aggiungano tutti i film\\ appena escono, vengono\\ circa 4 film al giorno.\end{tabular} \\ \hline
		11.2 &
		\begin{tabular}[c]{@{}c@{}}Aggiunta di persone che\\ hanno realizzato un film\\ alla piattaforma\end{tabular} &
		664 / mese &
		\begin{tabular}[c]{@{}c@{}}Considerando una stima\\ di 4 membri rilevanti del\\ cast di un film che si\\ vogliono tracciare, regista\\ compreso.\end{tabular} \\ \hline
	\end{longtable}
	\section{Raffinamento dello schema}
	\subsection{Eliminazione delle gerarchie}
	\subsubsection{	Gerarchia "Multimedia"}
	Questa gerarchia è stata risolta adottando il metodo del collasso verso il basso, questo perché l'accesso alle entità avviene separatamente (i film vengono reperiti in maniera separata dalle serie e viceversa). Questo metodo è inoltre applicabile perché la copertura della gerarchia è totale ed esclusiva.
	\begin{figure}[H]
		\centering
		\includegraphics{ER/ristrutturazione/ristmultimedia.png}
		\caption{Ristrutturazione della gerarchia "Multimedia"}
	\end{figure}
	\subsubsection{Gerarchia "Recensione"}
	Analogamente alla precedente, è stato usato il collasso verso il basso.
	\begin{figure}[H]
		\centering
		\includegraphics{ER/ristrutturazione/ristrecensione.png}
		\caption{Ristrutturazione della gerarchia "Recensione"}
	\end{figure}
	\subsubsection{Gerarchia "Membro Cast"}
	In questo caso abbiamo una copertura totale ma sovrapposta, e dato che in genere quando si consulta il cast di un film, che è una consultazione più frequente rispetto al reperimento di un singolo attore o regista, si accedono entrambi i tipi di membri del cast contemporaneamente, per questo si è optato per il collasso verso l'alto con selettori di tipo (copertura sovrapposta).
	\begin{figure}[H]
		\centering
		\includegraphics{ER/ristrutturazione/ristmembrocast.png}
		\caption{Ristrutturazione della gerarchia "Membro cast"}
	\end{figure}
	\subsubsection{Gerarchia "Template Promo"}
	Questa gerarchia è stata rielaborata mantenendo le entità e introducendo delle relazione al posto della gerarchia, questo perché vi è copertura totale ed esclusiva e ci sono relazioni distinte sia con il padre che con le entità figlie.
	\begin{figure}[H]
		\centering
		\includegraphics[width=400pt]{ER/ristrutturazione/ristpromo.png}
		\caption{Ristrutturazione della gerarchia "Template Promo"}
	\end{figure}
	\subsubsection{Gerarchia "Utilizzatore"}
	Analogamente alla precedente, è stato usato il mantenimento delle gerarchie (Se si fosse utilizzato il collasso verso il basso i nomi utente non sarebbero stati univoci tra i diversi tipi di utente).
	\begin{figure}[H]
		\centering
		\includegraphics{ER/ristrutturazione/ristutenza.png}
		\caption{Ristrutturazione della gerarchia "Utilizzatore"}
	\end{figure}
	\subsubsection{Schema completo dopo l'eliminazione delle gerarchie}
	\begin{figure}[H]
		\centering
		\includegraphics[width=450pt]{ER/ristrutturazione/ristcomp1.png}
		\caption{Schema ristrutturato completo, prima parte}
	\end{figure}
	\begin{figure}[H]
		\centering
		\includegraphics[width=450pt]{ER/ristrutturazione/ristcomp2.png}
		\caption{Schema ristrutturato completo, seconda parte}
	\end{figure}
	\section{Schemi di navigazione e tabelle degli accessi - Operazioni Utente}
	\subsection{Registrazione di un nuovo utente}
	\subsubsection{Schema di Navigazione}
	\begin{figure}[H]
		\centering
		\includegraphics[width=450pt]{ER/navigazione/registrazioneutente.png}
		\caption{Schema di navigazione - registrazione nuovo utente}
	\end{figure}
	\subsubsection{Tavola degli Accessi}
	\todo{TBD}
	
	\todo{Loss OP 2, 3.*}
	
	\subsection{Visualizzare tutto l'elenco dei film}
	\subsubsection{Schema di Navigazione}
	\begin{figure}[H]
		\centering
		\includegraphics[width=250pt]{ER/navigazione/visualizzarefilm.png}
		\caption{Schema di navigazione - visualizzare elenco film}
	\end{figure}
	\subsubsection{Tavola degli Accessi}
	\todo{TBD}
	\subsection{Visualizzare l’elenco dei film in base ad un’età scelta.}
	\subsubsection{Schema di Navigazione}
	\begin{figure}[H]
		\centering
		\includegraphics[width=250pt]{ER/navigazione/visualizzarefilm.png}
		\caption{Schema di navigazione - visualizzare elenco film in base all'età}
	\end{figure}
	\subsubsection{Tavola degli Accessi}
	\todo{TBD}
	\subsection{Contrassegnare come ”visualizzato” un film}
	\subsubsection{Schema di Navigazione}
	\begin{figure}[H]
		\centering
		\includegraphics[width=450pt]{ER/navigazione/visualizzatofilm.png}
		\caption{Schema di navigazione - contrassegnare un film "visualizzato"}
	\end{figure}
	\subsubsection{Tavola degli Accessi}
	\todo{TBD}
	\subsection{Recensire un film}
	\subsubsection{Schema di Navigazione}
	\begin{figure}[H]
		\centering
		\includegraphics[width=450pt]{ER/navigazione/recensionefilm.png}
		\caption{Schema di navigazione - recensire un film}
	\end{figure}
	\subsubsection{Tavola degli Accessi}
	\todo{TBD}
	\subsection{Visualizzare le recensioni di un film}
	\subsubsection{Schema di Navigazione}
	\begin{figure}[H]
		\centering
		\includegraphics[width=450pt]{ER/navigazione/visualrecensionifilm.png}
		\caption{Schema di navigazione - visualizzare le recensioni di un film}
	\end{figure}
	\subsubsection{Tavola degli Accessi}
	\todo{TBD}
	
	\todo{Loss 7.2}
	
	\subsection{Visualizzare una classifica dei generi più visualizzati}
	\subsubsection{Schema di Navigazione}
	\begin{figure}[H]
		\centering
		\includegraphics[width=450pt]{ER/navigazione/classificageneri.png}
		\caption{Schema di navigazione - visualizzare classifica dei generi più visualizzati}
	\end{figure}
	\subsubsection{Tavola degli Accessi}
	\todo{TBD}
	
	\todo{Loss 8.2}
	
	\section{Schemi di navigazione e tabelle degli accessi - Operazioni Amministratore}
	
	\todo{Loss 9.*, 10}
	
	\subsection{Aggiunta di un nuovo film alla piattaforma}
	\subsubsection{Schema di Navigazione}
	\begin{figure}[H]
		\centering
		\includegraphics[width=450pt]{ER/navigazione/aggiuntafilm.png}
		\caption{Schema di navigazione - aggiunta nuovo film alla piattaforma}
	\end{figure}
	\subsubsection{Tavola degli Accessi}
	\todo{TBD}
	\subsection{Aggiunta di persone che hanno realizzato un film alla piattaforma}
	\todo{TBMod OP}
	\subsubsection{Schema di Navigazione}
	\begin{figure}[H]
		\centering
		\includegraphics[width=450pt]{ER/navigazione/aggiuntacast.png}
		\caption{Schema di navigazione - aggiunta di persone che hanno realizzato un film}
	\end{figure}
	\subsubsection{Tavola degli Accessi}
	\todo{TBD}
	
	\section{Analisi delle ridondanze}
	\todo{TBD}
	\section{Traduzione di entità e associazioni in relazioni}
	\todo{TBD}
	\section{Schema relazionale finale}
	\todo{TBD}
	\section{Traduzione delle operazioni in query SQL}
	\todo{TBD}
	\chapter{Progettazione dell'applicazione}
	\todo{TBD}
\end{document}